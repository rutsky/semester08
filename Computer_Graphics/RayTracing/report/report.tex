% report.tex
% Report.
% Copyright (C) 2010 Vladimir Rutsky <altsysrq@gmail.com>

% TODO: Use styles according to GOST (it's hard).

\documentclass[a4paper,10pt]{article}

% Encoding support.
\usepackage{ucs}
\usepackage[utf8x]{inputenc}
\usepackage[T2A]{fontenc}
\usepackage[russian]{babel}

\usepackage{amsmath, amsthm, amssymb}

% Indenting first paragraph.
\usepackage{indentfirst}

\usepackage{url}
\usepackage[unicode]{hyperref}

%\usepackage[final]{pdfpages}

\usepackage[pdftex]{graphicx}

\newcommand{\HRule}{\rule{\linewidth}{0.5mm}}

% Spaces after commas.
\frenchspacing
% Minimal carrying number of characters,
\righthyphenmin=2

% From K.V.Voroncov Latex in samples, 2005.
\textheight=24cm   % text height
\textwidth=16cm    % text width.
\oddsidemargin=0pt % left side indention
\topmargin=-1.5cm  % top side indention.
\parindent=24pt    % paragraph indent
\parskip=0pt       % distance between paragraphs.
\tolerance=2000
%\flushbottom       % page height aligning
%\hoffset=0cm
%\pagestyle{empty}  % without numeration

\begin{document}

% Title page.
% title.tex
% Report title page.
% Copyright (C) 2010 Vladimir Rutsky <altsysrq@gmail.com>

\begin{titlepage} % начало титульной страницы

\begin{center} % включить выравнивание по центру

\large Санкт-Петербургский государственный политехнический университет\\[4.5cm]
% название института, затем отступ 4,5см

\huge Отчет по лабораторной работе \No 1\\[0.6cm] % название работы, затем отступ 0,6см
\large по~курсу <<Компьютерная графика>>\\[1cm]
\large <<Представление криволинейной поверхности координатной сеткой 
с удалением невидимых линий методом плавающего горизонта>>\\[3.7cm]
% тема работы, затем отступ 3,7см

\begin{flushright} % выровнять её содержимое по левому краю
\begin{tabular}{l l}
\emph{Студент:} & Руцкий~В.\,В.\\
\emph{Группа:} & 4057/2\\
\emph{Преподаватель:} & Ильин~Ю.\,П.
\end{tabular}
\end{flushright} % конец выравнивания по левому краю

\vfill % заполнить всё доступное ниже пространство

{\large Санкт-Петербург 2010}
\end{center} % закончить выравнивание по центру
\thispagestyle{empty} % не нумеровать страницу
\end{titlepage} % конец титульной страницы


% Content


\section{Задание}
Требуется модифицировать базовый пакет для создания фотореалистичных изображений трёхмерных сцен 
методом обратной трассировки лучей из \cite{shikin1995cg},
и добавить в него эффект дисперсии света при преломлении.

\section{Описание метода}
Рассмотрим физическую основу цвета.

Луч света представляет собой композицию волн разных длин.
Распределение количества волн по их длинам задаёт то, 
как этот луч света будет восприниматься наблюдателем, 
каким \textit{цветом} он будет видеть этот луч.
Диффузные или полупрозрачные объекты отражают или пропускают через себя волны лишь определённых длин,
в результате, наблюдатель видит объект определённого цвета, а не цвета источника света.

В используемой модели можно считать, 
что световые волны распространяются независимо друг от друга, 
поэтому можно разбить исходный диапазон длин волн источников 
на несколько непересекающихся групп диапазонов $g_1, \ldots, g_n$, 
осветить сцену по отдельности лучами из диапазонов разбиения~--- 
полученный набор изображений $I_1, \ldots, I_n$ будет содержать только цвета, 
соответствующие длинам волн $g_1, \ldots, g_n$, 
затем объеденить изображения $I_1, \ldots, I_n$ в результирующее изображение.

Дисперсия света~--- явление зависимости абсолютного показателя преломления вещества от длины волны света.
В результате дисперии света волны разной длины исходящие вдоль одного луча от источника света 
распространяются по разному.

Идея моделирования эффекта дисперсии света состоит в следующем.
Ограничим исходный набор длин волн источников трёмя составляющими $r$, $g$ и $b$~---
волнами, соответствующими красному, синему и зеленому цветам.
Построим три изображения сцены $I_r, I_g, I_b$, соответствующие
освещению только лучами с длинами волн взятых составляющих $r$, $g$ и $b$.
Так как изображения сцены $I_r, I_g, I_b$ строятся для определённых длин волн, 
при их построении для объектов сцены можно использовать показатели преломления соответствующие длинам
проходящих через них волн.
Тогда после объединения $I_r, I_g, I_b$ результирующее изображение будет учитывать дисперсию света для
основных составляющих $r$, $g$ и $b$.

\subsection{Детали реализации}
В исходной реализации метода трассировки лучей для каждого светопропускающего материала указывается 
величина абсолютного показателя преломления вещества, которая используется для рассчета угла преломления.

Для реализации эффекта дисперсии света, 
для каждого светопропускающего материала была указана величина абсолютного показателя преломления вещества для волн,
соответствующих взятым компонентам $r$, $g$ и $b$.
При первом преломлении производится разделение исходного луча на три луча, соответствующие $r$, $g$ и $b$.
При запуске луча, преломлённого полупрозрачным объектом, 
направление преломления выбирается в зависимости от обрабатываемой в данный момент компоненты.

Для увеличения реалистичности получаемых изображения был дополнительно реализован 
эффект полного внутреннего отражения.

\subsection{Проверка корректности}

\section{Результат работы}

\bibliographystyle{unsrt}
\bibliography{references.bib}

\end{document}
