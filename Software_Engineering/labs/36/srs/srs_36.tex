% srs_36.tex
% Software requirements specification for task #36.
% Vladimir Rutsky <altsysrq@gmail.com>
% 04.03.2010

% TODO: Use styles according to GOST (it's hard).

\documentclass[a4paper,10pt]{article}

% Encoding support.
\usepackage{ucs}
\usepackage[utf8x]{inputenc}
\usepackage[T2A]{fontenc}
\usepackage[russian]{babel}

%\usepackage{amsmath, amsthm, amssymb}

% Indenting first paragraph.
\usepackage{indentfirst}

%\usepackage{url}
%\usepackage{hyperref}

%\usepackage[final]{pdfpages}

% Spaces after commas.
\frenchspacing
% Minimal carrying number of characters,
\righthyphenmin=2

% From K.V.Voroncov Latex in samples, 2005.
\textheight=24cm   % text height
\textwidth=16cm    % text width.
\oddsidemargin=0pt % left side indention
\topmargin=-1.5cm  % top side indention.
\parindent=24pt    % paragraph indent
\parskip=0pt       % distance between paragraphs.
\tolerance=2000
%\flushbottom       % page height aligning
%\hoffset=0cm
%\pagestyle{empty}  % without numeration

% Using \paragraph as \subsubsubsection
\setcounter{secnumdepth}{4}
\setcounter{tocdepth}{4}

\title{Техническое задание на программу по ГОСТ 19.201-78}
\author{}
\date{}

\begin{document}

% Title page.
\maketitle

% Content

\section{Введение}
\subsection{Наименование программы}
Наименование~--- ``Инструмент преобразования кодировок''.
\subsection{Краткая характеристика области применения}
Программа предназначена для подбора исходной кодировки текста из интерфейса браузера Mozilla Firefox.

\section{Основания для разработки}
\subsection{Основание для проведения разработки}
Основанием для разработки является учебный план на весенний семестр 2010 года СПбГПУ. 
План согласован с деканатом ФМФ СПбГПУ и кафедрой «Прикладная математика», 
а также уполномоченным принимать задание Владиславом Арановым, 
именуемым в дальнейшем Заказчик, и утвержден студентом группы 4057/2 Владимиром Руцким, 
именуемым в дальнейшем Исполнитель, 19 февраля 2010.
\subsection{Наименование и условное обозначение темы разработки}
Наименование темы для разработки~---
``Разработка расширения для браузера Mozilla Firefox для автоматического подбора кодировки''.
Условное обозначение темы разработки (шифр темы)~--- ``Задача №36''.

\section{Назначение разработки}
\subsection{Функциональное назначение}
Функциональным назначением расширения является предоставление пользователю графического интерфейса 
для подбора кодировки отображаемой страницы.
\subsection{Эксплутационное назначение}
Программа должна эксплуатироваться на компьютерах учебгной лаборатории СПбГПУ.
Конечным пользователем программы должен являться непосредственно Заказчик.

\section{Требования к программе или программному изделию}
\subsection{Требования к функциональным характеристикам}
\subsubsection{Требования к составу выполняемых функций}
Расширение должно обеспечивать возможность выполнения следующих операций:
\begin{enumerate}
  \item преобразование текста из одной кодировки в другую,
  \item автоматический подбор кодировки на основе статистического анализа с использованием частотной таблицы.
\end{enumerate}
\subsubsection{Требования к организации входных данных}
Расширение должно предоставить возможность ввода текста следующими способами:
\begin{enumerate}
  \item ручной ввод,
  \item использование содержимого отображаемой страницы в качестве ввода.
\end{enumerate}

Расширение должно предоставить возможность замены встроенной частотной таблицы на пользовательскую.

\subsubsection{Требования к организации выходных данных}
Результат работы расширения должен быть доступен для копирования.

\subsubsection{Требования к временным характеристикам}
Требования к временным характеристикам расширения не предъявляются.

\subsection{Требования к надёжности}
\subsubsection{Требования к обеспечению надёжного (усточнивого) функционирования программы}
Надежное (устойчивое) функционирование программы должно быть обеспечено 
выполнением Заказчиком совокупности организационно-технических мероприятий, 
перечень которых приведен ниже:
\begin{enumerate}
  \item организацией бесперебойного питания технический средств,
  \item использованием лицензионного программного обеспечения.
\end{enumerate}

%\subsubsection{Время восстановления после отказа}
%\subsubsection{Отказы из-за некорректных действий оператора}
\subsection{Условия эксплуатации}
%\subsubsection{Климатические условия эксплуатации}
%\subsubsection{Требования к видам обслуживания}
\subsubsection{Требования к численности и квалификации персонала}
Конечный пользователь программы должен обладать базовыми навыками работы с компьютером.

\subsection{Требования к составу и параметрам технических средств}
В состав технических средств должен входить IBM-совместимый персональный компьютер 
с характеристиками, рекомендуемыми для работы с Mozilla Firefox.

\subsection{Требования к информационной и программной совместимости}
\subsubsection{Требования к информационным структурам и методам решения}
Метод решения задачи автоматического подбора кодировки должен включать частотный анализ.

\subsubsection{Требования к исходным кодам и языкам программирования}
Метод автоматического подбора кодировки должен быть реализован на языке программирования C++,
в составе динамически подгружаемой библиотеки.

\subsubsection{Требования к программным средствам, используемым программой}
Расширение должно работать на POSIX-совместимых платформах двоично-совместимых с компилятором GNU C++ версии не ниже 3.0.

%\subsubsection{Требования к защите информации и программ}
%\subsection{Требования к маркировке и упаковке}
%\subsubsection{Требования к маркировке}
%\subsubsection{Требования к упаковке}
%\paragraph{Условия упаковывания}
%\paragraph{Порядок упаковки}
%\subsection{Требования к транспортированию и хранению}
%\subsubsection{Условия транспортирования и хранения}
%\subsection{Специальные требования}

\section{Требования к программной документации}
Состав программной документации должен включать в себя:
\begin{enumerate}
  \item техническое задание,
  \item краткое руководство пользователя.
\end{enumerate}
%\subsection{Предварительный состав программной документации}

%\section{Технико-экономические показатели}
%\subsection{Экономические преимущества разработки}

%\section{Стадии и этапы разработки}
%\subsection{Стадии разработки}
%\subsection{Этапы разработки}
%\subsection{Содержание работ по этапам}

\section{Порядок контроля и приёмки}
\subsection{Виды испытаний}
Приемо-сдаточные испытания должны проводиться на кафедре СПбГПУ в сроки не более одного лабораторного занятия по курсу «Технология программирования».
Приемо-сдаточные испытания программы должны проводиться на двух наборах обоюдно согласованных тестов, 
подготовленных Заказчиком и Исполнителем, согласно Программе и методикам испытаний.
\subsection{Общие требования к приёмке работы}
На основании проведенных испытаний, Исполнитель ставит Заказчику зачет.

\section{Программа испытаний}
Испытания проводятся на ряде текстов умышленно представленных в неправильной кодировке.
В тесты включается:
\begin{itemize}
  \item проверка ручной последовательности перекодировки,
  \item проверка автоматического подбора кодировки.
\end{itemize}

\end{document}
